\documentclass[12pt]{article}
\linespread{1.5}

% PACKAGES
\usepackage[letterpaper,top=2cm,bottom=2cm,left=2cm,right=2cm]{geometry}
\usepackage[utf8]{inputenc}
\usepackage{amssymb}
\usepackage{enumitem}
\usepackage{amsmath}
\usepackage[mathscr]{eucal}
\usepackage[spanish, activeacute]{babel}
\usepackage{mathtools}
\usepackage{graphicx}
\usepackage{amsfonts}
\usepackage{float}
\usepackage{mathrsfs}
\usepackage{gauss}
\usepackage{multicol}
\renewcommand{\sin}{{\rm sen}\,}
\renewcommand{\sinh}{{\rm senh}\,}
\newcommand{\dst}{\displaystyle}

\title{Evaluación 3: Parte 1}
\date{2020 \\ Enero}
\author{Macedo Rodrigo \\Nombre \\ Nombre \\ Nombre \\ Nombre}

\begin{document}
\maketitle
\section{Ejercicio 1}
\par Mediante un análisis de frecuencias decifrar el siguiente texto que fue cifrado usando una traslación de la forma $C \equiv P + K$  mód 27: 
\begin{center}
    SIBMW ZPILM UCTMZ WAMAP TXWZB IUBMM UMSMA BCLPW LMSIK ZPXBW SWÑPI
\end{center}

\par Solución: \\
Lo primero que tenemos que revisar el el número de ocurrencias de cada letra dentro de la frase, así, haciendo un conteo obtenemos: 

\begin{center}
    \begin{tabular}{ |c|c|}
        \hline
            Letra & Repeticiones \\
        \hline
        M & 9 \\
        W & 5 \\
        B & 5 \\
        P & 4 \\
        Z & 4 \\
        I & 4 \\
        A & 3 \\
        U & 3 \\
        L & 3 \\
        S & 3 \\
        X & 2 \\ 
        T & 2 \\ 
        C & 2 \\ 
        K & 1 \\
        \hline
    \end{tabular}
\end{center}

Una vez obtenidos estos valores, podemos notar que la letra con mayor número de repeticiones es M con un total de 9. \\
Luego, con los valores obtenidos en clase sabemos que la letra que mas se repite dentro del lenguaje Español es la E y usamos sus respectivos valores dentro del abecedario.
\begin{equation}
    E = 4 \hspace{0.2cm} y \hspace{0.2cm} M = 12
\end{equation}

Como la M corresponde a la E, hacemos el desplazamiento $12 - 4 = 8$ y así nos queda la traslación utlizada para encriptar el mensaje es de la forma 
\begin{center}
    $C \equiv P + 8$ mód 27
\end{center} 

Y podemos obtener que la traslación que necesitamos para obtener el mensaje es:

\begin{center}
    $C \equiv P - 8$ mód 27
\end{center} 

la cual usaremos para obtener el mensaje deseado. \\

Ahora podemos construir todo nuestro abecedario con el desplazamiento obtenido 

\begin{center}
    \begin{tabular}{ |c|c|c|}
        \hline
            Alfabeto Original & Posición & Alfabeto trasladado \\
        \hline
        A & 0 & I \\
        B & 1 & J \\
        C & 2 & K \\
        D & 3 & L \\
        E & 4 & M \\
        F & 5 & N \\
        G & 6 & Ñ \\
        H & 7 & O \\
        I & 8 & P \\
        J & 9 & Q \\
        K & 10 & R \\ 
        L & 11 & S \\ 
        M & 12 & T \\ 
        N & 13 & U \\
        Ñ & 14 & V \\
        O & 15 & W \\
        P & 16 & X \\
        Q & 17 & Y \\
        R & 18 & Z \\
        S & 19 & A \\
        T & 20 & B \\
        U & 21 & C \\
        V & 22 & D \\
        W & 23 & E \\
        X & 24 & F \\
        Y & 25 & G \\
        Z & 26 & H \\
        \hline 
    \end{tabular}
\end{center}


\par Una vez obtenida la tabla anterior, podemos decifar el mensaje:
\begin{center}
    \textbf{“LA TEORIA DE NUMEROS ES IMPORTANTE EN EL ESTUDIO DE LA CRIPTOLOGÍA”}
\end{center}

\section{Ejercicio 2}

\par Madiante un análisis de frecuencias, desencriptar el siguiente texto que fue encriptado usando una transformación afín:
\begin{center}
    TFVS FMKK BUKB CKÑL BFSK MFGL KTFM CKUO ÑMFV DOBO KNMF VIII
\end{center}

\par Solución: \\
Realizando un análisis de frecuencias, observamos que cada letra tiene el siguiente número de apariciones en el text: 
\begin{center}
    \begin{tabular}{|c|c|}
        \hline
            Letra & Número de apariciones \\ 
        \hline
        G & 1 \\
        D & 1 \\
        N & 1 \\
        T & 2 \\
        S & 2 \\ 
        U & 2 \\ 
        C & 2 \\
        Ñ & 2 \\ 
        L & 2 \\
        V & 3 \\
        O & 3 \\
        I & 3 \\
        B & 4 \\
        M & 5 \\
        F & 7 \\
        K & 8 \\
        \hline
    \end{tabular}
\end{center}

Así, podemos observar que las letras que mas se repiten son la K con 8 apariciones y la F con 7. 

\par Con esta información podemos asociar a las letras del abecedario que mas se repiten, en este caso la E en primer lugar y en segundo lugar la A.

\begin{center}
    E = 4 $\rightarrow$ K = 10 \\ 
    A = 0 $\rightarrow$ F = 5
\end{center}

Y usando una transformación afín, obtenemos que 
\begin{center}
    $10 \equiv a*4 + b$ mód 27 \\
    $5 \equiv a*0 + b$ mód 27
\end{center}

Así, como $b \equiv 5$ mód 27, podemos asignar a la primera congruencia, obteniendo que 

\begin{center}
    $10 \equiv 4a + 5$ mód 27
\end{center}

Y lo único que haría falta encontrar sería el valor de ``a'' tal que $5 \equiv 4a$ mód 27. Si $a = 8$ entonces tenemos que 

\begin{center}
    $a \equiv 8$ mód 27 \\
    $b \equiv 5$ mód 27
\end{center}


Por lo que la transformación de encriptación es:
\begin{center}
    $C \equiv 8P + 5$ mód 27
\end{center}


Ahora, con esta información necesitamos encontrar una transformación para desencriptar el mensaje:

\begin{center}
    $P \equiv 8^{-1}(C - 5)$ mód 27 $\rightarrow$ $P \equiv 17(C - 5)$ mód 27
\end{center}


Siendo esta última nuestra transformación para desencriptar.

Así, obtenemos


\begin{center}
    \begin{tabular}{|c|c|c|c|}
        \hline
            Letra Cifrada & Posición en el alfabeto & Transformación aplicada & Letra descifrada \\
        \hline            
            T &  20 & 12 & M \\
            F &  5 & 0 & A \\
            V &  22 & 19 & S \\
            S &  19 & 22 & V \\
            M &  12 & 11 & L \\
            K &  10 & 4 & E \\
            B &  1 & 13 & N \\
            U &  21 & 2 & C \\
            C &  2 & 3 & D \\
            Ñ &  14 & 18 & R \\
            L &  11 & 21 & U \\
            G &  6 & 17 & Q \\
            O &  15 & 8 & I \\
            D &  3 & 20 & T \\
            N &  13 & 1 & B \\
            I &  8 & 24 & X \\
        \hline
    \end{tabular}
\end{center}


Y utilizando la tabla anterior, solos nos queda descifrar el texto, obteniendo 

\begin{center}
    \textbf{
        “MÁS VALE  ENCENDER UNA VELA QUE MALDECIR LAS TINIEBLAS XXX”
    }
\end{center}





\section{Ejercicio 3}
    ?`Qué transformación de cifrado se obtiene si se aplica la transformación $C \equiv 4P + 11$ mód 27 seguida de la transformación $C \equiv 10P + 20$ mód 27?
\end{document}
